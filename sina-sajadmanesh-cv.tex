%%%%%%%%%%%%%%%%%%%%%%%%%%%%%%%%%%%%%%%%%%%%%%%%%%%%%%%%%%%%%%%%%%%%%%%%
%%%%%%%%%%%%%%%%%%%%%% Simple LaTeX CV Template %%%%%%%%%%%%%%%%%%%%%%%%
%%%%%%%%%%%%%%%%%%%%%%%%%%%%%%%%%%%%%%%%%%%%%%%%%%%%%%%%%%%%%%%%%%%%%%%%

%%%%%%%%%%%%%%%%%%%%%%%%%%%%%%%%%%%%%%%%%%%%%%%%%%%%%%%%%%%%%%%%%%%%%%%%
%% NOTE: If you find that it says                                     %%
%%                                                                    %%
%%                           1 of ??                                  %%
%%                                                                    %%
%% at the bottom of your first page, this means that the AUX file     %%
%% was not available when you ran LaTeX on this source. Simply RERUN  %%
%% LaTeX to get the ``??'' replaced with the number of the last page  %%
%% of the document. The AUX file will be generated on the first run   %%
%% of LaTeX and used on the second run to fill in all of the          %%
%% references.                                                        %%
%%%%%%%%%%%%%%%%%%%%%%%%%%%%%%%%%%%%%%%%%%%%%%%%%%%%%%%%%%%%%%%%%%%%%%%%

%%%%%%%%%%%%%%%%%%%%%%%%%%%% Document Setup %%%%%%%%%%%%%%%%%%%%%%%%%%%%

% Don't like 10pt? Try 11pt or 12pt
\documentclass[9pt]{article}
\RequirePackage[T1]{fontenc}

% LaTeX will typeset using Computer Modern Roman, which a lot of
% non-mathematicians and non-engineers won't like. Also, a few PDF
% viewers may not render CMR very well. Instead, Times New Roman can
% be used. That's what this package does.
\usepackage{times}
\usepackage{fontawesome}
% The automated optical recognition software used to digitize resume
% information works best with fonts that do not have serifs. This
% command uses a sans serif font throughout. Uncomment both lines (or at
% least the second) to restore a Roman font (i.e., a font with serifs).
% (NOTE: This requires the times package above)
%\renewcommand{\familydefault}{\sfdefault}

% This is a helpful package that puts math inside length specifications
\usepackage{calc}

% This package helps LaTeX auto-hyphenate hyphenated words if you use
% special hyphens. For example, bio\-/mimicry will properly hyphenate
% ``mimicry'' if necessary.
\usepackage[shortcuts]{extdash}

% Layout: Puts the section titles on left side of page
\reversemarginpar

%
%         PAPER SIZE, PAGE NUMBER, AND DOCUMENT LAYOUT NOTES:
%
% The next \usepackage line changes the layout for CV style section
% headings as marginal notes. It also sets up the paper size as either
% letter or A4. By default, letter was used. If A4 paper is desired,
% comment out the letterpaper lines and uncomment the a4paper lines.
%
% As you can see, the margin widths and section title widths can be
% easily adjusted.
%
% ALSO: Notice that the includefoot option can be commented OUT in order
% to put the PAGE NUMBER *IN* the bottom margin. This will make the
% effective text area larger.
%
% IF YOU WISH TO REMOVE THE ``of LASTPAGE'' next to each page number,
% see the note about the +LP and -LP lines below. Comment out the +LP
% and uncomment the -LP.
%
% IF YOU WISH TO REMOVE PAGE NUMBERS, be sure that the includefoot line
% is uncommented and ALSO uncomment the \pagestyle{empty} a few lines
% below.
%

%% Use these lines for letter-sized paper
\usepackage[paper=a4paper,
            %includefoot, % Uncomment to put page number above margin
            marginparwidth=1in,     % Length of section titles
            marginparsep=.05in,       % Space between titles and text
            margin=0.5in,               % 1 inch margins
            includemp]{geometry}

%% Use these lines for A4-sized paper
%\usepackage[paper=a4paper,
%            %includefoot, % Uncomment to put page number above margin
%            marginparwidth=30.5mm,    % Length of section titlesa
%            marginparsep=1.5mm,       % Space between titles and text
%            margin=25mm,              % 25mm margins
%            includemp]{geometry}

%% More layout: Get rid of indenting throughout entire document
\setlength{\parindent}{0in}

% Provides special list environments and macros to create new ones
\usepackage[shortlabels]{enumitem}

% Simpler bibsections for CV sections
% (thanks to natbib for inspiration)
%
% * For lists of references with hanging indents and no numbers:
%
%   \begin{bibsection}
%       \item ...
%   \end{bibsection}
%
% * For numbered lists of references (with hanging indents):
%
%   \begin{bibenum}
%       \item ...
%   \end{bibenum}
%
%   Note that bibenum numbers continuously throughout. To reset the
%   counter, use
%
%   \restartlist{bibenum}
%
%   at the place where you want the numbering to reset.

\makeatletter
\newlength{\bibhang}
\setlength{\bibhang}{1em}
\newlength{\bibsep}
 {\@listi \global\bibsep\itemsep \global\advance\bibsep by\parsep}
\newlist{bibsection}{itemize}{3}
\setlist[bibsection]{label=,leftmargin=\bibhang,%
        itemindent=-\bibhang,
        itemsep=\bibsep,parsep=\z@,partopsep=0pt,
        topsep=0pt}
\newlist{bibenum}{enumerate}{3}
\setlist[bibenum]{label=[\arabic*],resume,leftmargin={\bibhang+\widthof{[999]}},%
        itemindent=-\bibhang,
        itemsep=\bibsep,parsep=\z@,partopsep=0pt,
        topsep=0pt}
\let\oldendbibenum\endbibenum
\def\endbibenum{\oldendbibenum\vspace{-.6\baselineskip}}
\let\oldendbibsection\endbibsection
\def\endbibsection{\oldendbibsection\vspace{-.6\baselineskip}}
\makeatother

%% Reference the last page in the page number
%
% NOTE: comment the +LP line and uncomment the -LP line to have page
%       numbers without the ``of ##'' last page reference)
%
% NOTE: uncomment the \pagestyle{empty} line to get rid of all page
%       numbers (make sure includefoot is commented out above)
%
\usepackage{fancyhdr,lastpage}
\pagestyle{fancy}
%\pagestyle{empty}      % Uncomment this to get rid of page numbers
\fancyhf{}\renewcommand{\headrulewidth}{0pt}
\fancyfootoffset{\marginparsep+\marginparwidth}
\newlength{\footpageshift}
\setlength{\footpageshift}
          {0.5\textwidth+0.5\marginparsep+0.5\marginparwidth-2in}
\lfoot{\hspace{\footpageshift}%
       \parbox{4in}{\, \hfill %
                    \arabic{page} of \protect\pageref*{LastPage} % +LP
%                    \arabic{page}                               % -LP
                    \hfill \,}}

% Finally, give us PDF bookmarks
\usepackage{color,hyperref}
\definecolor{darkblue}{rgb}{0.0,0.0,0.3}
\hypersetup{colorlinks,breaklinks,
            linkcolor=darkblue,urlcolor=darkblue,
            anchorcolor=darkblue,citecolor=darkblue}
\newcommand{\MYhref}[3][blue]{\href{#2}{\color{#1}{#3}}}%
%%%%%%%%%%%%%%%%%%%%%%%% End Document Setup %%%%%%%%%%%%%%%%%%%%%%%%%%%%


%%%%%%%%%%%%%%%%%%%%%%%%%%% Helper Commands %%%%%%%%%%%%%%%%%%%%%%%%%%%%

%%% HEADING AT TOP OF CURRICULUM VITAE

% The title (name) with a horizontal rule under it
% (optional argument typesets an object right-justified across from name
%  as well)
%
% Usage: \makeheading{name}
%        OR
%        \makeheading[right_object]{name}
%
% Place at top of document. It should be the first thing.
% If ``right_object'' is provided in the square-braced optional
% argument, it will be right justified on the same line as ``name'' at
% the top of the CV. For example:
%
%       \makeheading[\emph{Curriculum vitae}]{Your Name}
%
% will put an emphasized ``Curriculum vitae'' at the top of the document
% as a title. Likewise, a picture could be included:
%
%   \makeheading[{\includegraphics[height=1.5in]{my_picture}}]{Your Name}
%
% the picture will be flush right across from the name. For this example
% to work, make sure the extra set of curly braces is included. Also
% makes ure that \usepackage{graphicx} is somewhere in the preamble.
\newcommand{\makeheading}[2][]%
        {\hspace*{-\marginparsep minus \marginparwidth}%
         \begin{minipage}[t]{\textwidth+\marginparwidth+\marginparsep}%
             {\large \bfseries #2 \hfill #1}\\[-0.15\baselineskip]%
                 \rule{\columnwidth}{1pt}%
         \end{minipage}}

%%% SECTION HEADINGS

% The section headings. Flush left in small caps down pseudo-margin.
%
% Usage: \section{section name}
\renewcommand{\section}[1]{\pagebreak[3]%
    \vspace{1.3\baselineskip}%
    \phantomsection\addcontentsline{toc}{section}{#1}%
    \noindent\llap{\scshape\smash{\parbox[t]{\marginparwidth}{\hyphenpenalty=10000\raggedright #1}}}%
    \vspace{-\baselineskip}\par}

%%% LISTS

% This macro alters a list by removing some of the space that follows the list
% (is used by lists below)
\newcommand*\fixendlist[1]{%
    \expandafter\let\csname preFixEndListend#1\expandafter\endcsname\csname end#1\endcsname
    \expandafter\def\csname end#1\endcsname{\csname preFixEndListend#1\endcsname\vspace{-0.6\baselineskip}}}

% These macros help ensure that items in outer-type lists do not get
% separated from the next line by a page break
% (they are used by lists below)
\let\originalItem\item
\newcommand*\fixouterlist[1]{%
    \expandafter\let\csname preFixOuterList#1\expandafter\endcsname\csname #1\endcsname
    \expandafter\def\csname #1\endcsname{\let\oldItem\item\def\item{\pagebreak[2]\oldItem}\csname preFixOuterList#1\endcsname}
    \expandafter\let\csname preFixOuterListend#1\expandafter\endcsname\csname end#1\endcsname
    \expandafter\def\csname end#1\endcsname{\let\item\oldItem\csname preFixOuterListend#1\endcsname}}
\newcommand*\fixinnerlist[1]{%
    \expandafter\let\csname preFixInnerList#1\expandafter\endcsname\csname #1\endcsname
    \expandafter\def\csname #1\endcsname{\let\oldItem\item\let\item\originalItem\csname preFixInnerList#1\endcsname}
    \expandafter\let\csname preFixInnerListend#1\expandafter\endcsname\csname end#1\endcsname
    \expandafter\def\csname end#1\endcsname{\csname preFixInnerListend#1\endcsname\let\item\oldItem}}

% An itemize-style list with lots of space between items
%
% Usage:
%   \begin{outerlist}
%       \item ...    % (or \item[] for no bullet)
%   \end{outerlist}
\newlist{outerlist}{itemize}{3}
    \setlist[outerlist]{label=\enskip\textbullet,leftmargin=*}
    \fixendlist{outerlist}
    \fixouterlist{outerlist}

% An environment IDENTICAL to outerlist that has better pre-list spacing
% when used as the first thing in a \section
%
% Usage:
%   \begin{lonelist}
%       \item ...    % (or \item[] for no bullet)
%   \end{lonelist}
\newlist{lonelist}{itemize}{3}
    \setlist[lonelist]{label=\enskip\textbullet,leftmargin=*,partopsep=0pt,topsep=0pt}
    \fixendlist{lonelist}
    \fixouterlist{lonelist}

% An itemize-style list with little space between items
%
% Usage:
%   \begin{innerlist}
%       \item ...    % (or \item[] for no bullet)
%   \end{innerlist}
\newlist{innerlist}{itemize}{3}
    \setlist[innerlist]{label=\enskip\textbullet,leftmargin=*,parsep=0pt,itemsep=0pt,topsep=0pt,partopsep=0pt}
    \fixinnerlist{innerlist}

% An environment IDENTICAL to innerlist that has better pre-list spacing
% when used as the first thing in a \section
%
% Usage:
%   \begin{loneinnerlist}
%       \item ...    % (or \item[] for no bullet)
%   \end{loneinnerlist}
\newlist{loneinnerlist}{itemize}{3}
    \setlist[loneinnerlist]{label=\enskip\textbullet,leftmargin=*,parsep=0pt,itemsep=0pt,topsep=0pt,partopsep=0pt}
    \fixendlist{loneinnerlist}
    \fixinnerlist{loneinnerlist}

%%% EXTRA SPACE

% To add some paragraph space between lines.
% This also tells LaTeX to preferably break a page on one of these gaps
% if there is a needed pagebreak nearby.
\newcommand{\blankline}{\quad\pagebreak[3]}
\newcommand{\halfblankline}{\quad\vspace{-0.5\baselineskip}\pagebreak[3]}

%%% FORMATTING MACROS

% Provides a linked \doi{#1} that links doi:#1 to http://dx.doi.org/#1
\usepackage{doi}
% To change the text before the DOI, adjust this command
%\renewcommand\doitext{doi:}

% Provides a linked \url{#1} that doesn't require escape characters
\usepackage{url}

% You can adjust the style \url{} uses here:
% (options are: same, rm, sf, tt; defaults to tt)
\urlstyle{same}

% For \email{ADDRESS}, links ADDRESS to the url mailto:ADDRESS
% (uncomment to typeset the e\-/mail address in typewriter font;
%  otherwise, will be typeset in the \urlstyle above)
%\DeclareUrlCommand\emaillink{\urlstyle{tt}}
\providecommand*\emaillink[1]{\nolinkurl{#1}}
\providecommand*\email[1]{\href{mailto:#1}{\emaillink{#1}}}

\providecommand\BibTeX{{B\kern-.05em{\sc i\kern-.025em b}\kern-.08em \TeX}}
\providecommand\Matlab{\textsc{Matlab}}

% Custom hyphenation rules for words that LaTeX has trouble with
\hyphenation{bio-mim-ic-ry bio-in-spi-ra-tion re-us-a-ble pro-vid-er Media-Wiki}

%%%%%%%%%%%%%%%%%%%%%%%% End Helper Commands %%%%%%%%%%%%%%%%%%%%%%%%%%%

%%%%%%%%%%%%%%%%%%%%%%%%% Begin CV Document %%%%%%%%%%%%%%%%%%%%%%%%%%%%

\begin{document}
\makeheading{Sina Sajadmanesh}

\section{Contact Information}

% NOTE: Mind where the & separators and \\ breaks are in the following
%       table. Table is one row made up of three parboxes. The left
%       parbox has address info, the middle parbox has a vertical bar,
%       and the right parbox has phone and electronic contact
%       information.
%
% MACROS: \rcollength is the width of the right column of the table
%             (adjust it to your liking; default is 1.85in).
%         \spacewidth is width of area between left and right boxes.
%
%\newlength{\rcollength}\setlength{\rcollength}{0in}%
\newlength{\spacewidth}\setlength{\spacewidth}{20pt}
%
%\begin{tabular}[t]{@{}p{\textwidth-\rcollength-\spacewidth}@{}p{\spacewidth}@{}p{\rcollength}}%

{Rue Marconi 19 \hfill (+41) 27-721-77-58~\faPhone\\
1920 Martigny \hfill{\href{mailto:sina.sajadmanesh@idiap.ch}{sina.sajadmanesh@idiap.ch}~\faEnvelope} \\ 
Switzerland \hfill{\href{https://www.idiap.ch/~sajadmanesh}{https://www.idiap.ch/\string~sajadmanesh}~\faHome}}

%\hfill{GitHub: \href{http://www.github.com/pauljwright}{www.github.com/pauljwright}} }


%\end{tabular}

%%
%% In modern CV's, it seems like ``Objective'' is frowned upon. Instead,
%% incorporate it into a well-constructed cover letter. The ``More
%% information'' can go at the end of the CV, but it should not distract
%% from the section giving references available to contact.
%%
%
% \section{Objective}
%
% Placement in an academic position (i.e., faculty, postdoctoral, or
% research scientist) that allows for advanced research in distributed
% complex adaptive systems (i.e., modeling, analysis, design, and
% verification) with a particular focus on the control of engineered
% agents (e.g., for communications, control, software, electronics, and
% sustainability) and the analysis of biological phenomena (e.g.,
% self-organization, ecological rationality)
% \begin{innerlist}
% \item More information and auxiliary documents can be found at\\\url{http://www.tedpavlic.com/facjobsearch/}
% \end{innerlist}

\section{Research Summary}

My research interests lie at the intersection of privacy, deep learning, and graph analysis. More specifically, I use privacy enhancing technologies, such as differential privacy and federated learning, with graph representation learning algorithms, including graph neural networks, to make them more private, secure, and robust for real-world applications.


%%%% EDUCATION %%%%
\section{Education}

\href{https://www.epfl.ch/en/home/}{\textbf{École Polytechnique Fédérale de Lausanne (EPFL)}}, Lausanne, Switzerland \hfill {May 2019 -- May 2023}
\begin{innerlist}
\item[] Ph.D. in Electrical Engineering \quad GPA: 5.75 / 6
        \begin{innerlist}
        \item[] \textbf{Thesis Topic:} \emph{Learning over Graphs: From Social to Privacy-Preserving Methods}
        \item[] \textbf{Adviser:} Prof.~Daniel Gatica-Perez
        \item[] \textbf{Relevant Courses:} Artificial Neural Networks (Deep Reinforcement Learning), Deep Learning for Natural Language Processing
        \end{innerlist}

\end{innerlist}

\halfblankline

\href{http://www.en.sharif.edu/}{\textbf{Sharif University of Technology}}, Tehran, Iran \hfill {Sep 2014 -- Sep 2016}
\begin{innerlist}
	\item[] M.Sc. in Information Technology Engineering \quad  GPA: 18.1 / 20
	\begin{innerlist}
		\item[] \textbf{Thesis Topic:} \emph{Link Prediction in Heterogeneous Multi-Layer Social Networks}
		\item[] \textbf{Adviser:} Prof.~Hamid R. Rabiee
		\item[] \textbf{Relevant Courses:} Machine Learning, Complex Dynamical Networks, Performance Modeling of Computer Systems, Advanced Network Security, Database Security and Privacy
	\end{innerlist}
	
\end{innerlist}

\halfblankline

\href{http://ui.ac.ir/EN}{\textbf{University of Isfahan}}, Esfahan, Iran \hfill {Sep 2009 -- Feb 2014}
\begin{innerlist}
	\item[] B.Sc. in Computer Software Engineering \quad  GPA: 16.19 / 20 (Last four semesters: 17.4 / 20)
	\begin{innerlist}
		\item[] \textbf{Project:} \emph{Design and Implementation of an Android App for Voice Control of Household Devices}
		\item[] \textbf{Adviser:} Prof.~Ahamd R. Naghsh-Nilchi
		\item[] \textbf{Relevant Courses:} Data Structures, Algorithms, Probability and Statistics, Artificial Intelligence, Information Retrieval, Software Engineering, Databases, Operating Systems, Computer Networks
	\end{innerlist}
	
\end{innerlist}

\section{Research Experience}


\textbf{Research Assistant} \hfill {May 2019 -- present}
\begin{innerlist}
    \item[] \href{https://www.idiap.ch/en/scientific-research/social-computing/index_html}{Social Computing Group}, \textbf{\href{https://idiap.ch}{Idiap Research Institute}}, Martigny, Switzerland
    \begin{innerlist}
    	\item Developing privacy-preserving graph neural network models using differential privacy to reduce the privacy risks of using graph representation learning algorithms in real applications.
    \end{innerlist}
\end{innerlist}

\halfblankline

\textbf{Research Assistant} \hfill {Nov 2014 -- May 2019}
\begin{innerlist}
	\item[] \href{http://dml.ce.sharif.edu/}{Data Science and Machine Learning Lab}, \href{http://www.en.sharif.edu/}{\textbf{Sharif University of Technology}}, Tehran, Iran
	\begin{innerlist}
		\item Private Deep Learning: Worked on a hybrid approach based on Siamese fine-tuning and split learning to make non-private pre-trained deep learning models privacy-preserving at the inference stage.
		\item Food and Cuisines: Analyzed a large-scale collection of recipes published on the web and their content, aiming to understand cuisines
		and culinary habits around the world.
		\item Social Media Mining: Developed time-aware link prediction algorithms over heterogeneous social networks using recurrent neural networks and non-parametric machine learning.
	\end{innerlist}
\end{innerlist}

\section{Teaching Experience}

\textbf{Guest Lecturer} \hfill {Fall 2017}
\begin{innerlist}
	\item[] \href{http://ce.sharif.edu/}{Department of Computer Engineering}, \href{http://www.en.sharif.edu/}{\textbf{Sharif University of Technology}}, Tehran, Iran
	\begin{innerlist}
		\item[] \textbf{Course:} Fundamentals of Programming (Python)
		\item[] \textbf{Website:} \href{http://ce.sharif.edu/courses/96-97/1/ce153-12/}{http://ce.sharif.edu/courses/96-97/1/ce153-12/}
	\end{innerlist}
\end{innerlist}

\halfblankline

\textbf{Teaching Assistant}
\begin{innerlist}
	\item[] \href{http://ce.sharif.edu/}{Department of Computer Engineering}, \href{http://www.en.sharif.edu/}{\textbf{Sharif University of Technology}}, Tehran, Iran
	\begin{innerlist}
		\item Artificial Intelligence (Head TA) \hfill Spring 2017
		\item Advanced Topics in Artificial Intelligence - Statistical Learning Theory \hfill Spring 2016
		\item Engineering Probability and Statistics \hfill Spring 2016
	\end{innerlist}
%	\halfblankline
	\item[] \href{https://comp.ui.ac.ir/en}{Faculty of Computer Engineering}, \href{http://ui.ac.ir/EN}{\textbf{University of Isfahan}}, Esfahan, Iran
	\begin{innerlist}
		\item Artificial Intelligence \hfill Fall 2013
		\item Advanced Computer Programming 2 - JavaFx and Android \hfill Fall 2012
		\item Computer Programming - Java \hfill Fall 2011
		\item Computer Programming - C++ \hfill Fall 2010
	\end{innerlist}
\end{innerlist}

\newpage
\makeheading{Sina Sajadmanesh} 

\section{Industrial Experience}

\textbf{Big-Data Engineer} \hfill {Sep 2018 -- May 2019}
\begin{innerlist}
	\item[] \href{http://ictic.sharif.ir}{Sharif ICT Innovation Center}, Tehran, Iran
	\begin{innerlist}
		\item Responsible for building a native big-data processing platform using state of the art technologies, such as Spark, Cassandra, JanusGraph, Elasticsearch, etc.
	\end{innerlist}
\end{innerlist}

\halfblankline

\textbf{Software Engineer Intern} \hfill {Summer 2012}
\begin{innerlist}
	\item[] Amin Computer Co., Esfahan, Iran
	\begin{innerlist}
		\item Responsible for designing and developing an Android application for company's web-based human resource management system.
	\end{innerlist}
\end{innerlist}

 
%
% % Add a little space to nudge next ``Ref'd Journal Publications'' marginpar
% % down to make room for tall ``Submitted Journal Publications''
% % marginpar. If there are enough submitted journal publications, this
% % space will not be needed (and should be removed).
% \vspace{0.1in}

\section{Publications}

\begin{bibenum}
	\item{} \textbf{Sina Sajadmanesh} and Daniel Gatica-Perez \hfill  Oct 2020\\
	\href{https://arxiv.org/abs/2006.05535}{\textbf{Locally Private Graph Neural Networks}}\\
	\textit{Technical Report, ArXiv e-prints}
	
	\item{} Seyed Ali Osia, Ali Shahin Shamsabadi, \textbf{Sina Sajadmanesh}, \textit{et al} \hfill May 2020\\
	\href{https://arxiv.org/abs/1703.02952}{\textbf{A Hybrid Deep Learning Architecture for Privacy-Preserving Mobile Analytics}}\\
	\textit{IEEE Internet of Things Journal}
	
	\item{} \textbf{Sina Sajadmanesh}, Sogol Bazargani, Jiawei Zhang, and Hamid R. Rabiee \hfill Aug 2019\\
	\href{https://arxiv.org/abs/1710.00818}{\textbf{Continuous-Time Relationship Prediction in Dynamic Heterogeneous Information Networks}}\\
	\textit{ACM Transactions on Knowledge Discovery from Data}
	
	\item{} \textbf{Sina Sajadmanesh}, Jiawei Zhang, and Hamid R. Rabiee \hfill Jun 2017\\
	\href{https://arxiv.org/abs/1706.06783}{\textbf{NPGLM: A Non-Parametric Method for Temporal Link Prediction}}\\
	\textit{Technical Report, ArXiv e-prints}
	
	\item{} \textbf{Sina Sajadmanesh}, Sina Jafarzadeh, Seyed Ali Ossia, \textit{et al} \hfill Apr 2017\\
	\href{https://arxiv.org/pdf/1610.08469}{\textbf{Kissing Cuisines: Exploring Worldwide Culinary Habits on the Web}}\\
	International World Wide Web Conference (WWW 2017) Companion
	
	\item{} \textbf{Sina Sajadmanesh}, Hamid R. Rabiee and Ali Khodadadi \hfill  Aug 2016\\
	\href{https://arxiv.org/pdf/1607.08821}{\textbf{Predicting Anchor Links between Heterogeneous Social Networks}}\\
	\textit{IEEE/ACM International Conference on Advances in Social Networks Analysis and Mining}
	
\end{bibenum}


\section{Honors and Awards}
\begin{innerlist}
\item
\textbf{PhD admission}
{with fully-funded research assistantship to the Computer Science program from the University of Illinois at Urbana-Champaign.}, 
{2018}

\item
\textbf{PhD admission}
{with fully-funded postgraduate studentship to the Computer Science and Engineering program from Hong-Kong University of Science and Technology.},
{2017}

\item
\textbf{Ranked 3rd}
{in cumulative GPA among B.Sc. Computer Software Engineering students admitted for Fall 2009, University of Isfahan
},
{2014}

\item
\textbf{Ranked 6th}
{in Iranian nationwide university entrance exam for graduate studies, field
	of Artificial Intelligence, among more than 100000 students},
{2014}

\item
\textbf{Ranked 15th}
{in Iranian nationwide university entrance exam for graduate studies, field
	of Computer Networks and Security, among more than 30000 students},
{2014}

\item
\textbf{Ranked 28th}
{in 18th National Computer Olympiad for University Students at Tarbiat Modares University}
, {Tehran, Iran}, 
{2013}

\item
\textbf{Ranked 16th}
{in ACM-ICPC regional contest, Asia region, Tehran site, among more than 70 teams at University of Tehran}
, {Tehran, Iran},
{2011}

\item
\textbf{Ranked 2nd}
{in nationwide collegiate programming contest among more than 70 teams at University of Kashan}
, {Kashan, Iran},
{2010}

\item
\textbf{Ranked among top 0.02\%}
{in Iran's nationwide university entrance exam for undergraduate studies.},
{2009}
\end{innerlist}


\section{Memberships}
     \textbf{ACM-ICPC Student Chapter}, \href{http://ui.ac.ir/EN}{{University of Isfahan}}, Esfahan, Iran \hfill {2010 -- 2012}
    \begin{innerlist}
     \item Organized weekly programming contests.
     \item Instructed data structures and algorithms to freshmen.
    \end{innerlist}
    
\section{Community Service}

\textbf{Journal Reviewer}
\begin{innerlist}
	\item[] \href{https://www.springer.com/journal/11280}{World Wide Web} (2018)
	\item[] \href{https://www.springer.com/journal/13278}{Social Network Analysis and Mining} (2020)
	\item[] \href{https://dl.acm.org/journal/tist}{ACM Transactions on Intelligent Systems and Technology} (2020)
\end{innerlist}

\newpage
\makeheading{Sina Sajadmanesh} 

\section{Media Coverage}
\begin{innerlist}
	\item {\href{https://www.technologyreview.com/s/602790/how-data-mining-reveals-the-worlds-healthiest-cuisines/}{How Data Mining Reveals the World’s Healthiest Cuisines}}, \textbf{MIT Technology Review}, {3 Nov 2016}
	
	\item {\href{https://www.indy100.com/article/healthy-diverse-top-healthiest-countries-cuisine-food-in-the-world-list-7412171}{These are the world's most diverse cuisines}}, \textbf
	{The Independent}, {11 Nov 2016}
	
	\item {\href{https://www.france24.com/fr/20161115-algorithme-compare-cuisines-monde-matiere-dingredients-dapports-nutritionnels}{Un algorithme compare les cuisines du monde en matière d'ingrédients et d'apports nutritionnels}}, \textbf
	{France 24}, {15 Nov 2016}
	
	\item {\href{https://reachmd.com/news/if-you-are-what-you-eat-regional-cuisines-have-a-major-impact-on-health/1306703/}{If you are what you eat: regional cuisines have a major impact on health}}, \textbf
	{ReachMD}, {4 Nov 2016}
	
	\item {\href{https://www.sciencesetavenir.fr/high-tech/data/diversite-nutrition-les-cuisines-du-monde-analysees-par-les-big-data_108012}{Les cuisines du monde passées au crible des big data}}, \textbf
	{Sciences et Avenir}, {14 Nov 2016}
\end{innerlist}


\section{Technical Skills} 

\textit
{Programming:}\\
{Python, Java, C, C++, MATLAB, PHP, Javascript}

\halfblankline

\textit
{Information Retrieval \& Analytics:}\\
{Elasticsearch, JanusGraph, Cassandra}

\halfblankline

\textit
{Data Science and Machine Learning:}\\
{PyTorch, PyTorch-Geometric, PyTorch-Lightning, Scikit-Learn, Pandas}



\section{References}
\textbf{Prof. Daniel Gatica-Perez}
\begin{innerlist}
\item[] Idiap Research Institute, EPFL
\item[] Website: \href{https://idiap.ch/~gatica}{https://idiap.ch/\string~gatica}
\item[] Email: \href{mailto:daniel.gatica-perez@epfl.ch}{daniel.gatica-perez@epfl.ch}
\end{innerlist}    

\halfblankline

\textbf{Prof. Hamid R. Rabiee}
\begin{innerlist}
	\item[] Sharif University of Technology
	\item[] Website: \href{http://sharif.ir/~rabiee}{http://sharif.ir/\string~rabiee}
	\item[] Email: \href{mailto:rabiee@sharif.edu}{rabiee@sharif.edu}
\end{innerlist}   

\halfblankline

\textbf{Prof. Emiliano De Cristofaro}
\begin{innerlist}
	\item[] University College London
	\item[] Website: \href{https://emilianodc.com/}{https://emilianodc.com/}
	\item[] Email: \href{mailto:e.decristofaro@ucl.ac.uk}{e.decristofaro@ucl.ac.uk}
\end{innerlist}

\halfblankline

\textbf{Prof. Hamed Haddadi}
\begin{innerlist}
	\item[] Imperial College London
	\item[] Website: \href{https://haddadi.github.io/}{https://haddadi.github.io/}
	\item[] Email: \href{mailto:h.haddadi@imperial.ac.uk}{h.haddadi@imperial.ac.uk}
\end{innerlist} 

\end{document}

%%%%%%%%%%%%%%%%%%%%%%%%%% End CV Document %%%%%%%%%%%%%%%%%%%%%%%%%%%%%

%----------------------------------------------------------------------%
% The following is copyright and licensing information for
% redistribution of this LaTeX source code; it also includes a liability
% statement. If this source code is not being redistributed to others,
% it may be omitted. It has no effect on the function of the above code.
%----------------------------------------------------------------------%
% Copyright (c) 2007, 2008, 2009, 2010, 2011 by Theodore P. Pavlic
%
% Unless otherwise expressly stated, this work is licensed under the
% Creative Commons Attribution-Noncommercial 3.0 United States License. To
% view a copy of this license, visit
% http://creativecommons.org/licenses/by-nc/3.0/us/ or send a letter to
% Creative Commons, 171 Second Street, Suite 300, San Francisco,
% California, 94105, USA.
%
% THE SOFTWARE IS PROVIDED "AS IS", WITHOUT WARRANTY OF ANY KIND, EXPRESS
% OR IMPLIED, INCLUDING BUT NOT LIMITED TO THE WARRANTIES OF
% MERCHANTABILITY, FITNESS FOR A PARTICULAR PURPOSE AND NONINFRINGEMENT.
% IN NO EVENT SHALL THE AUTHORS OR COPYRIGHT HOLDERS BE LIABLE FOR ANY
% CLAIM, DAMAGES OR OTHER LIABILITY, WHETHER IN AN ACTION OF CONTRACT,
% TORT OR OTHERWISE, ARISING FROM, OUT OF OR IN CONNECTION WITH THE
% SOFTWARE OR THE USE OR OTHER DEALINGS IN THE SOFTWARE.
%----------------------------------------------------------------------%
